\documentclass[a4paper, 12pt]{article}
\usepackage[utf8]{inputenc}
\usepackage[T1]{fontenc}
\usepackage[danish]{babel}
\title{
    En Kasse I En Festforening\\
    Regelsæt
}
\author{
    Formanden E18, @lkymia\\
    BFORM\\
    GINKA\\
}
\date{13. november 2018}
\begin{document}
\maketitle
\begin{enumerate}
 \item Et hold består af 6 personer, bestående af mindst ét styks bestyrelsesmedlem og ét styks pige.
 \item Hvert hold skal drikke 30 stks. alm. pilsnerøl.
 \item Øllene må først åbnes i auditoriet på dagen.
 \item Hvert hold stiller med to slattællere og to tidstagere.
 \item Producerer et hold 30 cL slat udløser dette 2 straføl.
 \item Producerer et hold yderligere slat, straffes der 1 øl pr. 3 cL over 30 cL.
	 \subitem Eksempel 1: 30 cL øl udløser 2 straføl
	 \subitem Eksempel 2: [30-33] cL udløser 2 straføl
	 \subitem Eksempel 3: [33-36] cL udløser 3 straføl
	 \subitem Eksempel 4: [36-39] cL udløser 4 straføl
 \item Slat fra straføl tælles med og kan derfor potientielt udløse yderligere straføl.
 \item Hvis et hold ikke har medbragt nok øl til deres straføl, har de tabt.
 \item På det holdets bord skal der skal være en tapestreg langs midten af bordet. Spillere skal placere færdige øl bag tapestregen.
 \item Vælger en spiller at ørle er spilleren ude. Spilleren sørger selv for at dømme om høn ørler.
 \item Vælger spilleren at ørle i en øl, udskiftes øllen med en ny øl.
 \item Vælger spilleren at vælte en øl, og der kommer øl ud, udløser det en straføl. Den væltede øl tæller ikke med som slat.
 \item Det er ikke i spillets ånd at producere mere end 30 cL slat.
 \item Øl slatter afvejes med vægt
\end{enumerate}
\end{document}
